\section{Running Anything on the BEAM}

There are a number of possible targets if we would like to run
anything on the infamous BEAM. For starters, there is \emph{Erlang
source code} and \emph{BEAM bytecode}. Neither of these have
up-to-date specification documents\footnotemark, let alone any formal
specifications, beyond the Erlang/OTP implementation itself.

\footnotetext{The last Erlang language specification was written up in
1999\cite{erlang:1999:spec}. It attained only a draft form, and is
today considered rather outdated. See also
\url{https://github.com/erlang/spec}.}

On its way from Erlang source code to BEAM bytecode, the Erlang/OTP
compiler takes a valid Erlang program through three additional
intermediate, serializable representations. Such a happy path is
illustrated in Figure \ref{fig:serializable-representations}; the
intermediate representations are \emph{Erlang Abstract Format},
\emph{Core Erlang}, and \emph{BEAM assembly}.  They are
``serializable'' in the sense that Erlang/OTP can both produce and
consume files in these intermediate formats.

\NewDocumentCommand\optimizing{}{$\circlearrowright$}

\begin{figure}[h]
\centering
\tikzstyle{block} = [rectangle, draw, 
    text width=4.5em, text centered, rounded corners,
    node distance=7em,
    minimum height=4em]
\tikzstyle{line} = [draw, -latex']
\tikzstyle{optimizing} = [line width=0.2em]

\begin{tikzpicture}[auto]
\node (source) [block] {Erlang source code};
\node (absform) [block, right of=source] {Erlang Abstract Format};
\node (core) [block, right of=absform] {Core Erlang\\ \optimizing{}};
\node (asm) [block, right of=core] {BEAM assembly\\ \optimizing{}};
\node (bin) [block, right of=asm] {BEAM bytecode};
\path [line] (source) -> (absform);
\path [line] (absform) -> (core);
\path [line] (core) -> (asm);
\path [line] (asm) -> (bin);
\end{tikzpicture}

\caption{Serializable representations along the path of a valid
program through the passes of the Erlang/OTP compiler. The symbol
\optimizing{} marks the formats on which the compiler performs crucial
optimisations.}

\label{fig:serializable-representations}
\end{figure}

Generating code in any of the formats listed in Figure
\ref{fig:serializable-representations}, would allow us to run code on
the BEAM, by leveraging the Erlang/OTP infrastructure.

However, the Erlang/OTP compiler is an optimising compiler. Various
optimisations are performed on both the Core Erlang representation,
and the BEAM assembly representation. Optimisations take place in two
stages because the two representations suitable for each their own
class of optimisations. In particular, Core Erlang is a piecewise
functional language, while BEAM assembly is inherently imperative.

% Probably the best-specified and most stable of these is Core
% Erlang\cite{erlang:erlang:2001:core-intro, erlang:2004:core-spec}.

% Erlang The BEAM Book suggests that ``Core Erlang is the best target
% for a language you want to run in ERTS''. The reasons mentioned are
% that Core Erlang is less subject to change than the languages of
% BEAM and Erlang.
