\section{Running Anything on the BEAM}

There are a number of possible targets if we would like to run
anything on the infamous BEAM. For starters, there is \emph{Erlang
source code} and \emph{BEAM bytecode}. Neither of these have
up-to-date specification documents\footnotemark, let alone any formal
specifications, beyond the Erlang/OTP implementation itself.

\footnotetext{The last Erlang language specification was written up in
1999\cite{erlang:1999:spec}. It attained only a draft form, and is
today considered rather outdated. See also
\url{https://github.com/erlang/spec}.}

On its way from Erlang source code to BEAM bytecode, the Erlang/OTP
compiler takes a valid Erlang program through two additional
intermediate, serializable representations. Such a happy path is
illustrated in Figure \ref{fig:serializable-representations}; the
intermediate representations are called \emph{Erlang Abstract Format}
and \emph{Core Erlang}.  They are ``serializable'' in the sense that
Erlang/OTP can both produce and consume files in these intermediate
formats.
 
\begin{figure}[h]
\centering
\tikzstyle{block} = [rectangle, draw, 
    text width=4.5em, text centered, rounded corners,
    node distance=8em,
    minimum height=4em]
\tikzstyle{line} = [draw, -latex']

\begin{tikzpicture}[auto]
\node (source) [block] {Erlang source code};
\node (absform) [block, right of=source] {Erlang Abstract Format};
\node (core) [block, right of=absform] {Core Erlang};
\node (beam) [block, right of=core] {BEAM bytecode};
\path [line] (source) -> (absform);
\path [line] (absform) -> (core);
\path [line] (core) -> (beam);
\end{tikzpicture}

\caption{Serializable representations along the path of a valid
program through the passes of the Erlang/OTP compiler.}

\label{fig:serializable-representations}
\end{figure}

Generating code in any of the formats listed in Figure
\ref{fig:serializable-representations}, would allow us to run code on
the BEAM, by leveraging the Erlang/OTP infrastructure.

The Erlang/OTP compiler is a heavily optimising compiler. These
optimisations are primarily done on Core Erlang code, and rely heavily
on the functional nature of Core Erlang. BEAM bytecode however, is
inherently imperative, and is not further optimised at compile time.

% Probably the best-specified and most stable of these is Core
% Erlang\cite{erlang:erlang:2001:core-intro, erlang:2004:core-spec}.

% Erlang The BEAM Book suggests that ``Core Erlang is the best target
% for a language you want to run in ERTS''. The reasons mentioned are
% that Core Erlang is less subject to change than the languages of
% BEAM and Erlang.
