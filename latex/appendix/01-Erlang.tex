\chapter{Erlang}
\label{app:erlang}

\section{On the Popularity of Erlang}
\label{app:popularity-of-erlang}

\section{An Overview of the Erlang/OTP Source Code}
\label{app:overview-of-erlang-source}

\subsection{Erlang/OTP is Open Source}

Erlang/OTP is an open-source project in the following sense:

\begin{itemize}

\item The source code is readily available at either:

\begin{itemize}

\item \url{https://erlang.org/downloads/}

\item \url{https://github.com/erlang/otp/}

\end{itemize}

\item The code is published under an open source license: Apache
License 2.0.

\item There are various direct communication channels with the
Erlang/OTP developers: there is a Slack team for Erlang, an IRC
channel on FreeNode (\texttt{\#erlang}), as well as various mailing
lists under \texttt{erlang.org}:

\begin{itemize}

\item \url{http://erlang.org/mailman/listinfo/erlang-questions}

The main forum for Erlang/OTP questions and discussions.

\item \url{http://erlang.org/mailman/listinfo/eeps}

Discussions wrt. specific Erlang Enhancement Proposals (EEPs).

\end{itemize}

\item The developers maintain a public issue tracker:

\url{https://bugs.erlang.org}

Community members can go here to find outstanding problems, post any
bugs they may discover, and glean at some, but far from all of the
internal development process.

\item There are contribution guidelines for the project. These require
for community contributions to come in as Pull Requests on GitHub:

\url{https://github.com/erlang/otp/blob/master/CONTRIBUTING.md}

\end{itemize}
