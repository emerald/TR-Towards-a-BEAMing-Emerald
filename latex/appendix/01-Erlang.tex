\chapter{Erlang}
\label{app:erlang}

\section{On the Popularity of Erlang}
\label{app:popularity-of-erlang}

Erlang seems like a fairly popular programming language. To back up
that claim with data, we can take a look at the TIOBE Index, and try
to gauge the Erlang community.

\subsection{The TIOBE Index}

The TIOBE Index is compiled by the TIOBE Company, a company that works
to asses and track the quality of software\footnote{See also
\url{https://www.tiobe.com/company/about/}.}. The index has been
compiled since June 2001, on a monthly basis. The results have been
available at the URL \url{https://www.tiobe.com/tiobe-index/} since at
least July 2016. Raw data since 2001 is available for an additional
fee.

Figure \longref{fig:erlang-tiobe-top-50} shows the positioning of
Erlang among the top-50 programming languages, according to the TIOBE
Index, since July 2016. Erlang has stayed within the top-100
throughout this period. However, it is struggling to stay in the
top-50 as of late.

\begin{figure}[h!]
\centering
\includegraphics[width=0.8\textwidth]{appendix/01-Erlang/tiobe-top-50-plot.eps}
\caption{Erlang's positioning among the top-50 programming languages
according to the TIOBE Index in the period July 2016 to October 2019.
Source: Internet Archive versions of the URL
\url{https://www.tiobe.com/tiobe-index/}.}
\label{fig:erlang-tiobe-top-50}
\end{figure}

\subsection{The ACM SIGPLAN Erlang Workshop}

This workshop has seen 18 installments, occurring annually since
2002\footnote{See also
\url{https://www.erlang.org/community/workshops}.}. The very first
Erlang Workshop however, was held in 2001, but does not count among
the 18, as it was not an ACM SIGPLAN workshop.

The workshop has always accepted no more than a dozen papers, in the
format of extended abstracts or full research papers, limited to 12
pages in length.  Coupled with 1-2 invited talks, the workshop has
never spanned more than a day.

The workshop was first co-located with PLI (Principles, Logics, and
Implementations of High-Level Programming Languages), but since 2004,
it has been co-located with ICFP (International Conference on
Functional Programming).

\subsection{Other Erlang Conferences \& Events}

The other Erlang conferences \& events have had a perhaps less
academic, and more industrial focus. Erlang experts come here to
inspire developers with talks, trainings, and workshops.

\begin{description}

\item[Erlang User Conference]\ \\ Organized annually in Stockholm
between 1994 and 2017.

Now renamed to Code BEAM STO (under Code Sync Conferences).

\item[Erlang Factory]\ \\ Organized annually between 2008 and 2017 in
various cities around the world, including Palo Alto, San Francisco,
London, and Berlin.

Now rebranded as Code Sync Conferences.

\item[Code Sync Conferences]\ \\ The above conferences \& events
combined with similar events for the Elixir programming language, and
other smaller events since 2018. Held in major cities around the
world.

See more at \url{https://www.codesync.global/}.

\end{description}

\section{Erlang/OTP is Open Source / Open to Collaboration}
\label{app:erlang-otp-is-open-source}

Although many software projects claim to be ``Open Source'', there are
often crucial differences in-how-far the lead developers of are open
to collaboration with the community at large. Erlang/OTP can well be
considered ``open to collaboration'', at least in the following
senses:

\begin{itemize}

\item The source code is readily available at either:

\begin{itemize}

\item \url{https://erlang.org/downloads/}

\item \url{https://github.com/erlang/otp/}

\end{itemize}

\item The code is published under an Open Source license:

\begin{itemize}

\item Apache License 2.0 (since OTP 18.0, released 24 July, 2015)

\end{itemize}

\item Erlang/OTP is fairly well-documented:

\begin{itemize}

\item \url{https://www.erlang.org/docs}

\end{itemize}

\item There are various direct communication channels with core
Erlang/OTP developers and enthusiasts: there is a Slack team for
Erlang, an IRC channel on FreeNode (\texttt{\#erlang}), as well as
various mailing lists; for instance:

\begin{itemize}

\item \url{http://erlang.org/mailman/listinfo/erlang-questions}

The main forum for Erlang/OTP questions and discussions.

\end{itemize}

\item The core Erlang/OTP developers maintain a public issue tracker:

\begin{itemize}

\item \url{https://bugs.erlang.org}

\end{itemize}

Community members can go here to find some outstanding problems, post
the bugs they discover, and follow along as bugs get fixed.

\item There are contribution guidelines for Erlang/OTP. For instance,
these require for community contributions to come in as Pull Requests
on GitHub:

\begin{itemize}

\item \url{https://github.com/erlang/otp/blob/master/CONTRIBUTING.md}

\end{itemize}

\item There is a well-defined process for adding enhancements and
extensions to Erlang. This ensures that proposals and decisions are
well-documented in a way that with the intent of documenting proposals
and decisions.

This is called the Erlang Enhancement Process\footnote{See more at
\url{https://www.erlang.org/erlang-enhancement-proposals}.}, which
demands that significant changes are described Erlang Extension
Proposals (EEPs), posted on GitHub\footnote{In particular, on
\url{https://github.com/erlang/eep}.}, and discussed on a designated
mailing-list:

\begin{itemize}

\item \url{http://erlang.org/mailman/listinfo/eeps}

\end{itemize}

\end{itemize}

\section{An Overview of the Erlang/OTP Source Code}
\label{app:overview-of-erlang-source}

\begin{description}

\item[\texttt{lib/compiler/src/compile.erl}]~

This module defines the Erlang compiler interface, as an Erlang
module. The C module \texttt{erts/common/erlc.c} declares the actual
entry-point for \texttt{erlc}, and interfaces with the above Erlang
module via the \texttt{erts} C API.

\item[\texttt{lib/compiler/src/cerl.erl}]~

This module declares Core Erlang, an intermediate representation
designed explicitly to be easy to optimise. It has a simpler syntax
than standard Erlang, but retains its functional nature.

\item[\texttt{lib/compiler/src/genop.tab}]~

This file defines the generic BEAM instruction set. These definitions
are used by both the compiler, and the runtime system. They are
written in a small domain-specific language, and there is a Perl
script\footnote{See \texttt{erts/emulator/utils/beam\_makeops}.}, to
generate both Erlang code for compiler, and C code for the runtime,
from this file, coupled with other \texttt{.tab} files.

\end{description}
