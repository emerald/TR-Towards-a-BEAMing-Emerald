\chapter{Erlang}
\label{app:erlang}

\section{On the Popularity of Erlang}
\label{app:popularity-of-erlang}

\section{An Overview of the Erlang/OTP Source Code}
\label{app:overview-of-erlang-source}

\subsection{Erlang/OTP is Open Source / Open to Collaboration}

Although many software projects claim to be ``Open Source'', there are
often crucial differences in-how-far the lead developers of are open
to collaboration with the community at large. Erlang/OTP can well be
considered ``open to collaboration'', at least in the following
senses:

\begin{itemize}

\item The source code is readily available at either:

\begin{itemize}

\item \url{https://erlang.org/downloads/}

\item \url{https://github.com/erlang/otp/}

\end{itemize}

\item The code is published under an Open Source license:

\begin{itemize}

\item Apache License 2.0 (since OTP 18.0, released 24 July, 2015)

\end{itemize}

\item Erlang/OTP is fairly well-documented:

\begin{itemize}

\item \url{https://www.erlang.org/docs}

\end{itemize}

\item There are various direct communication channels with core
Erlang/OTP developers and enthusiasts: there is a Slack team for
Erlang, an IRC channel on FreeNode (\texttt{\#erlang}), as well as
various mailing lists; for instance:

\begin{itemize}

\item \url{http://erlang.org/mailman/listinfo/erlang-questions}

The main forum for Erlang/OTP questions and discussions.

\end{itemize}

\item The core Erlang/OTP developers maintain a public issue tracker:

\begin{itemize}

\item \url{https://bugs.erlang.org}

\end{itemize}

Community members can go here to find some outstanding problems, post
the bugs they discover, and follow along as bugs get fixed.

\item There are contribution guidelines for Erlang/OTP. For instance,
these require for community contributions to come in as Pull Requests
on GitHub:

\begin{itemize}

\item \url{https://github.com/erlang/otp/blob/master/CONTRIBUTING.md}

\end{itemize}

\item There is a well-defined process for adding enhancements and
extensions to Erlang. This ensures that proposals and decisions are
well-documented in a way that with the intent of documenting proposals
and decisions.

This is called the Erlang Enhancement Process\footnote{See more at
\url{https://www.erlang.org/erlang-enhancement-proposals}.}, which
demands that significant changes are described Erlang Extension
Proposals (EEPs), posted on GitHub\footnote{In particular, on
\url{https://github.com/erlang/eep}.}, and discussed on a designated
mailing-list:

\begin{itemize}

\item \url{http://erlang.org/mailman/listinfo/eeps}

\end{itemize}

\end{itemize}
