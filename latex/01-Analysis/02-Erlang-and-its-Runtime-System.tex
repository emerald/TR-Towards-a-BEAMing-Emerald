\section{Erlang and its Runtime System}

Erlang is a general-purpose programming language, designed in the late
1980s to early 1990s, in an industrial effort to improve the
development environment for telephone switches. It was designed
internally at Ericsson (a Swedish networking and telecommunications
company), and open-sourced in 1998. Erlang boasts features in support
for programming highly concurrent, distributed, fault-ridden systems.
According to the 2019 TIOBE
Index\footnote{\url{https://web.archive.org/web/20190416025331/https://www.tiobe.com/tiobe-index/}},
Erlang is today among the top-50 most popular programming languages.

Erlang is a dynamically-typed, compiled, actor-based language. Erlang
programs are composed of ``actors'' --- concurrently executing
light-weight processes, that communicate with each other (almost)
exclusively using messages.

The Erlang runtime system was built to support ``concurrency,
distribution, and
fault-tolerance''\footnote{\url{https://web.archive.org/web/20190414185439/http://erlang.org/}}.
