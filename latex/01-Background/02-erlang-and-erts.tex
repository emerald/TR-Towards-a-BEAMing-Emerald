\chapter{Erlang and the Erlang Runtime System(s)}

Erlang is a general-purpose programming language, designed in the late
1980s to early 1990s, in an industrial effort to improve the
development environment for telephone switches\cite{erlang:hopl:2007}.
It was designed internally at Ericsson (a networking and
telecommunications company), and open-sourced in 1998. Erlang boasts
features in support for programming highly concurrent, distributed,
and faulty systems, which nevertheless are intended to ``run
forever''.

According to the April 2019 TIOBE
Index\footnote{\url{https://web.archive.org/web/20190416025331/https://www.tiobe.com/tiobe-index/}},
Erlang is today the 50th most popular programming language. It is in
use not only at Ericsson\cite{media:2018-03-31:Erlang-20-years}, but
also at
Klarna\cite{media:2015-01-09:Klarna-Engineering-Insights}\footnote{See
also \url{https://github.com/klarna?language=erlang}} (an online
payments processor),
Riak\footnote{\url{https://github.com/basho/riak}} (an
enterprise-grade, distributed NoSQL database), and perhaps more
notoriously, at Facebook\cite{media:2008-03-13:facebook-chat}, and its
subsidiary, WhatsApp\cite{media:2014-02-21:Inside-Erlang}.

Erlang is a dynamically-typed, compiled, actor-based
language\cite{erlang:2019:Erlang-OTP}. It is ``actor-based'' in the
sense that an Erlang program is composed of \emph{actors} ---
concurrently executing light-weight processes, that communicate with
each other (almost) exclusively using messages.  Although the actions
that an actor should take in response to a message are specified in a
\emph{functional} way, it is hard to call Erlang a functional
programming language --- the lifetime of an actor is governed by the
messages it receives, and these are both dynamically typed, and arrive
in ill-predictable order. Both of these shortcomings however, can be
partially addressed through an additional static analyser, called
Dialyzer\cite{erlang:2019:Dialyzer}.

Unlike in an ``object-based'' system, message-sends are asynchronous,
and the messages received by an actor are first collected in the
actor's \emph{inbox}. An actor can then inspect the messages in the
inbox in a particular order: \emph{oldest message first}.  However,
the actor may skip over messages it is not yet ready handle.  It is
the programmer's responsibility to ensure that such messages are
eventually handled; Erlang (without Dialyzer) does not enforce this.

In terms of distribution, Erlang also boasts ``fine-grained mobility''
and ``location transparency'', but in a slightly different way: (1)
users must manage their messages \emph{explicitly}, and (2) it is the
location of an actor, not the location of an ``object'', which is
transparent --- sending a message to a remote actor is syntactically
indistinguishable from sending a message to a local actor.

The Erlang runtime system has seen several
implementations\cite{erlang:2018:The-Erlang-Runtime-System}. The
de-facto standard implementation today goes by the name Erlang/OTP,
and is maintained in an open-source fashion\footnote{See also Appendix
\longref{app:erlang-otp-is-open-source}.}, supported by Ericsson.
Erlang/OTP features a virtual machine called BEAM for executing
Erlang. Despite Erlang's popularity, and despite an open-source
development process, the internals of Erlang/OTP, and BEAM in
particular, remain poorly documented. The best that I could find, is
an incomplete community effort, separate from the development of
Erlang/OTP itself: \url{https://github.com/happi/theBeamBook}.

\section{The Erlang Programming Model}

% Perhaps Methodology, rather than Model?

% The basic compilation unit in Erlang is a module. A module has a
% name; it declares a number of functions, some of which may be
% exported.

In the Erlang world-view, a distributed system is a universe of
\emph{actors}, segregated among a number of \emph{nodes}. An Erlang
node is an instance of the Erlang \emph{runtime system}, available at
a particular network address. New nodes join the universe, by
addressing some node already in it.

Erlang nodes run on real hardware, in the context of real operating
systems, and communicate over real networks. All these may fail
unexpectedly, and Erlang nodes may temporarily, or permanently become
\emph{unavailable}. The Erlang programming model attempts to
accommodate Erlang programmers accordingly (more on this below).

An Erlang actor is a light-weight Erlang \emph{process}, executing a
sequence of Erlang \emph{functions}. An Erlang function is written in
functional programming style, and a function can terminate by
tail-recursively calling another function. An Erlang actor/process
terminates when no such tail-recursive call is made.

Furthermore, an Erlang actor maintains an \emph{inbox} for messages
from other actors. Any actor can send any kind of message to any other
actor they have a reference to (Erlang is dynamically typed). An actor
processes its inbox at its leisure, at its own pace. Fundamentally, an
Erlang actor executes a sequence of Erlang \emph{operations}, which
can be segregated as follows:

\begin{description}

\item[\textbf{basic}]\ \\ Execute a built-in function / basic
operation

\item[\textbf{call}]\ \\ Call a programmer-specified function and wait
for the result

\item[\textbf{tailcall}]\ \\ Pass control to another
programmer-specified function

\item[\textbf{spawn\_local}]\ \\ Spawn a new process (actor) to execute
a given function

\item[\textbf{spawn\_remote}]\ \\ Spawn a new process (actor) at
another node to execute a given function

\item[\textbf{send}]\ \\ Send a message to some actor's inbox (can be
own inbox)

\item[\textbf{receive}]\ \\ Wait for, and take out the first matching
message from the inbox

\end{description}

\subsection{Receiving Messages}

It is notable that actors are not interrupted upon the arrival of a
message. They must explicitly issue \textbf{receive} operations, when
ready. Hence, messages will build up in the inbox if an actor doesn't
regularly empty it. Luckily, there isn't much an actor can do without
interacting with other actors.

Furthermore, less important messages can build up in the inbox, while
an actor is working to respond to more important messages. In
particular, the \textbf{receive} operation does not simply take out
the top message, but iterates over the messages, in order of their
arrival, and takes out the first matching message, once such a message
arrives. A \textbf{receive} operation is a \emph{blocking} operation,
but can be suspended after a given timeout.

It follows, that the programmer has good control over how the inbox is
processed, with a few notable exceptions: \begin{inparaenum}\item[(a)]
an actor has exactly one inbox; and \item[(b)] messages are processed
in order of arrival\end{inparaenum}.  Hence, to implement an efficient
prioritization mechanism, where less important messages are not
repeatedly inspected, programmers must employ multiple cooperating
actors.

\subsection{Sending Messages}

To \textbf{send} a message to an actor, we must hold a reference to it.
The \textbf{spawn}-operations yield such references, and they can be
passed around like any other datatype. Perhaps surprisingly, received
messages are not decorated with a reference to the sender. If the
sender expects a reply, the sender must pack a reference to itself (or
another actor), inside the message.

A \textbf{send} operation is \emph{non-blocking} --- the Erlang
runtime system does not wait around for the message to be delivered,
or wait around for a reply. If an actor has something to say, it packs
it into a message, issues a \textbf{send} operation, and can goes on
to do more important things. If instead the actor expects a subsequent
reply, it must employ subsequent \textbf{receive} operations.

\subsection{Fault Tolerance}

By default, the sender is not notified in any way about what happens
with the receiver. However, a connection between two actors can be
established, such that one is notified when the other becomes
unavailable (i.e., the process terminates, the host Erlang runtime
system stops, or the connection is lost).

When an actor \textbf{receive}s a notification of another's death, it
may respond as it sees fit. This connection can be established when
spawning an actor, using variants of the \textbf{spawn}-operations, as
well as some other \textbf{basic} operations; neither are listed here
for brevity.

\subsection{Residency and Concurrency}

Erlang actors are fixed at the nodes where they are spawned. However,
they can communicate with actors resident on other nodes. There is no
syntactic difference between communicating with a local, and a remote
actor. The only syntactic difference lies in how such actors are
spawned\footnote{Vis-a-vis the \textbf{spawn} operations above.}.

It is safe to say that Erlang is location transparent. However, since
actors do not move, it is also easy to segregate local and remote
actors, and one can keep their location in mind when communicating
with them.

In support of the location transparency illusion (among other reason),
actors that reside on the same node, use separate address spaces. The
only way to share data in Erlang, is by message-passing. Effectively,
this also eliminates the possibility of race conditions in general,
but does add some overhead to sharing data locally.

\subsection{Dynamic Typing}

Since message passing is a basic communication primitive, and any
actor can send any message to any other actor, messages carry their
type at runtime. This entails runtime costs, and code may fail at
runtime due to ill-formed messages. However, this does offer ease of
entry-level programming and flexibility.

Although there are some tools for the static analysis of Erlang
programs, to find type-errors before they happen at runtime, these
tools do not strip the runtime types. Hence, Erlang in general, is
dynamically typed.

