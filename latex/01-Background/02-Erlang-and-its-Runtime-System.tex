\section{Erlang and its Runtime System(s)}

Erlang is a general-purpose programming language, designed in the late
1980s to early 1990s, in an industrial effort to improve the
development environment for telephone switches\cite{erlang:hopl:2007}.
It was designed internally at Ericsson (a networking and
telecommunications company), and open-sourced in 1998. Erlang boasts
features in support for programming highly concurrent, distributed,
and faulty systems, which nevertheless are intended to ``run
forever''.

According to the April 2019 TIOBE
Index\footnote{\url{https://web.archive.org/web/20190416025331/https://www.tiobe.com/tiobe-index/}},
Erlang is today the 50th most popular programming language. It is in
use not only at Ericsson\cite{media:2018-03-31:Erlang-20-years}, but
also at
Klarna\cite{media:2015-01-09:Klarna-Engineering-Insights}\footnote{See
also \url{https://github.com/klarna?language=erlang}} (an online
payments processor),
Riak\footnote{\url{https://github.com/basho/riak}} (an
enterprise-grade, distributed NoSQL database), and perhaps more
notoriously, at Facebook\cite{media:2008-03-13:facebook-chat}, and its
subsidiary, WhatsApp\cite{media:2014-02-21:Inside-Erlang}.

Erlang is a dynamically-typed, compiled, actor-based
language\cite{erlang:2019:Erlang-OTP}. It is ``actor-based'' in the
sense that an Erlang program is composed of \emph{actors} ---
concurrently executing light-weight processes, that communicate with
each other (almost) exclusively using messages.  Although the actions
that an actor should take in response to a message are specified in a
\emph{functional} way, it is hard to call Erlang a functional
programming language --- the lifetime of an actor is governed by the
messages it receives, and these are both dynamically typed, and arrive
in ill-predictable order. Both of these shortcomings however, can be
partially addressed through an additional static analyser, called
Dialyzer\cite{erlang:2019:Dialyzer}.

Unlike in an ``object-based'' system, message-sends are asynchronous,
and the messages received by an actor are first collected in the
actor's \emph{inbox}. An actor can then inspect the messages in the
inbox in a particular order: \emph{oldest message first}.  However,
the actor may skip over messages it is not yet ready handle.  It is
the programmer's responsibility to ensure that such messages are
eventually handled; Erlang (without Dialyzer) does not enforce this.

In terms of distribution, Erlang also boasts ``fine-grained mobility''
and ``location transparency'', but in a slightly different way: (1)
users must manage their messages \emph{explicitly}, and (2) it is the
location of an actor, not the location of an ``object'', which is
transparent --- sending a message to a remote actor is syntactically
indistinguishable from sending a message to a local actor.

The Erlang runtime system has seen several
implementations\cite{erlang:2018:The-Erlang-Runtime-System}. The
de-facto standard implementation today goes by the name Erlang/OTP,
and is maintained in an open-source fashion\footnote{See also Appendix
\longref{app:erlang-otp-is-open-source}.}, supported by Ericsson.
Erlang/OTP features a virtual machine called BEAM for executing
Erlang. Despite Erlang's popularity, and despite an open-source
development process, the internals of Erlang/OTP, and BEAM in
particular, remain poorly documented. The best that I could find, is
an incomplete community effort, separate from the development of
Erlang/OTP itself: \url{https://github.com/happi/theBeamBook}.
