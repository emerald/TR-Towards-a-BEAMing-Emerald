\chapter{Emerald and the Emerald VM}

Emerald is a general-purpose programming language, designed in the
early to mid-1980s, in an academic effort to address the challenge of
programming distributed systems. This effort culminated in numerous
academic publications\cite{emerald:tse:1987, emerald:tocs:1988,
emerald:spe:1991} and PhD theses\cite{emerald:phd:norman-c-hutchinson,
emerald:phd:eric-jul, emerald:phd:niels-christian-juul}. However,
Emerald has gained little ground as a programming language --- it
remains only narrowly used in small, academic circles. As such, the
elements of the Emerald programming environment are only piecewise
efficient, robust, and well-documented.

Emerald is a statically-typed, compiled, object-based
language\cite{emerald:tse:1987, emerald:tocs:1988, emerald:spe:1991}.
It is ``object-based'' (and not ``object-oriented'') in the sense that
(almost) everything is an object (as in Smalltalk), and it uses
\emph{structural subtyping} (as in Go), rather than \emph{nominal
subtyping} (as in Java or C\#). All communication in Emerald happens
through method invocation, although objects may perform work
concurrently on one, or many different machines. The type-system also
boasts parametric polymorphism, and a hint of (compiled away)
dependent types.

The key distribution features of Emerald are \emph{fine-grained
mobility} and \emph{location transparency}: (1) anything, from values
of basic types to large objects, can be moved around; and (2) the
location of an object is transparent to the programmer --- a remote
method invocation is syntactically indistinguishable from a local
method invocation.

Emerald programs are compiled into bytecode for an Emerald-specific
virtual machine (also called the
``kernel''\cite{emerald:phd:eric-jul}). The Emerald VM was crafted
around the same time as Emerald itself.  Unlike the Emerald
programming model, it has received less academic scrutiny. Chances
are, the Emerald VM is severely outdated, and it is no match for a
more modern distributed VM.

This claim however, warrants further investigation. After-all, Emerald
and the Emerald VM were specifically designed to address some the
performance shortcomings of the distributed systems programming
languages at the time\cite{emerald:phd:eric-jul}. Specifically,
Emerald attempted to keep execution of local code efficient, while
sacrificing the efficiency of mobility. The reason being that network
operations take such a long time anyway, that mobility could take the
extra costs. This has not changed. Network operations are still the
ones that take the longest. However, what is locally efficient,
robust, and considered secure has changed. Especially in the face of
more heterogeneous systems.

\section{The Emerald Programming Model}

In the Emerald world-view, a distributed system is a universe of
\emph{objects}, distributed across a number of \emph{nodes}. An
Emerald node is an instance of the Emerald \emph{runtime}, available
at a particular network address. New nodes join the universe, by
addressing some node already in it.

Due to the nature of distributed systems however, nodes, and so the
objects they house, may temporarily, or permanently become
\emph{unavailable}, at any given time. Emerald nodes run on real
hardware, in the context of real operating systems, and communicate
over real networks. All these may misbehave unexpectedly, and the
Emerald programming model attempts to accommodate Emerald programmers
accordingly (more on this further below).

Other than housing objects, an Emerald node may also be running a
number of concurrent Emerald \emph{processes}. An Emerald process
executes a program-defined sequence of Emerald \emph{operations},
which can be fundamentally segregated as follows:

\begin{enumerate}[label=(\arabic*)]

\item Execute a built-in function / basic operation
\label{enum:op:builtin}

\item Invoke a method on an object (local or remote)
\label{enum:op:invoke}

\end{enumerate}

Despite a clear runtime difference between remote, and local method
invocation, Emerald does not draw a syntactic difference between the
two. To this effect, Emerald source code is location transparent, and
it is up to the Emerald runtime(s) to deliver on this syntactical
illusion.

The body of a method (local or remote) is again a sequence of
operations of the kinds (1) and (2). As in many other languages,
operations of kind \ref{enum:op:builtin} perform basic computational
tasks, and interact with the surrounding operating system. However,
there are also some notable basic operations in Emerald:

\begin{enumerate}[label=(1\alph*)]

\item List currently available nodes

\item Suggest that an object (local or remote) is moved to a specific
node \label{enum:op:move}

\item Try and fix an object at a specific node \label{enum:op:fix}

\item Try and unfix an object, so that it can move about again
\label{enum:op:unfix}

\item Instantiate a new object at the current node \label{enum:op:new}

\end{enumerate}

\subsection{Mobility}

Objects in the Emerald universe are not (initially) fixed at the nodes
where they are instantiated --- they can roam about the Emerald
universe, throughout their lifetime. Programmers can exert some
control over their movement, but mostly, they can merely guide it ---
it is up to the Emerald runtime(s) to actually decide what moves,
where, and when.

In particular, operations of kind \ref{enum:op:move} provide a
\emph{hint} to the Emerald runtime, that it might be a good idea to
move an object there and then. Operations of kind \ref{enum:op:fix}
mandate that an object is moved to a given node, that instant, and
that it stays there until it is ``unfixed''. However, the object is
only moved (and fixed), provided that the target node is still
available, and that the move happens within reasonable time.
Similarly, an operation of kind \ref{enum:op:unfix} succeeds only if
the given object (i.e., the node it is resident on) is still
available.

The possibility of failure that plagues operations of kinds
\ref{enum:op:fix} and \ref{enum:op:unfix}, also plagues operations of
kind \ref{enum:op:invoke}: A remote method invocation may fail due to
the target object being unavailable. The lack of a syntactic
distinction between remote and local method invocations, and lax
control over object movement means that method invocation, without
further care, may fail unexpectedly.

\subsection{Fault Tolerance}

In distributed systems, all sorts of things can fail, and our programs
must be prepared. To this end, the Emerald programming model banks on
an \emph{unavailability}-handling mechanism. This works similar to
exception-handling in non-distributed programming models: The
programmer can wrap a sequence of Emerald operations into a block, and
attach an unavailability handler to it. If any of the operations fail
due to the unavailability of a node (or object), control is first
handed over to the unavailability handler, and then outside the block.

\subsection{Processes and Their Mobility}

Operations of kind \ref{enum:op:new} are notable because they may
spawn new processes.

There are two kinds of processes in Emerald: \emph{start-} and
\emph{spawned} processes. A start-process starts when a new node is
started, with a specific Emerald compilation unit to execute. It stops
after having executed the operations mandated by the compilation unit.
The operations of a spawned process are declared together with an
object type, and a process is spawned whenever an object of that type
is instantiated.  This process is then tied to the said object. If the
object moves, so does the process. Hence, both objects and (spawned)
processes can roam about in a universe of Emerald nodes.

\subsection{Fine-Grained Mobility}

As with ``objects'' in other languages, Emerald objects are first and
foremost structures of data.  Furthermore, as in only \emph{some}
other languages with ``objects'', in Emerald (almost) everything is an
object. Hence, an Emerald object is really more a structure of
references (to other objects), than a structure of data.

By default, when this ``structure of references'' moves, it is up to
the run-time to decide whether the objects it refers to move with it.
However, references can be explicitly marked as \emph{attached}, and
attached objects always move together with the object(s) they're
attached to.

Attaching references can be important to the correct and efficient
functioning of an object. Attached objects always reside on the same
node, and so working with only attached objects can never lead to
remote method invocations. Such code has predictable performance and
cannot fail due to the unavailability of some other node.

% The only thing that can happen in such a start-process is that a
% number of objects are instantiated, and assigned process-local
% names. Objects instantiated later, can then refer to objects
% instantiated earlier in the start-process. More interestingly
% however, as part of the instantiation of any object in Emerald, a
% new process may be spawned.  Hence, while perhaps starting with a
% single start-process, an Emerald node can go on to execute many
% processes \emph{concurrently}.

% If stated with a start-process, a node stays alive long enough to
% execute all of its effective processes. If started without a
% start-process, the node stays alive indefinitely, perpetually
% awaiting work from other nodes. Such ``work'' is sent along by
% moving around Emerald objects, and their associated processes.

% An object is always instantiated at a particular node, but can roam
% about, throughout its lifetime; and die anywhere. The process,
% possibly instantiated as part of the initialization of an object, is
% tied to this object. If the object moves, so does the process.

% An Emerald object is alive so long as some Emerald process,
% somewhere, retains a reference to it. For these, effectively
% free-range, Emerald objects to communicate, they do not need to
% explicitly keep track of each other's location --- this is done
% automatically by the Emerald runtimes involved. This makes Emerald
% code \emph{location-transparent}.

